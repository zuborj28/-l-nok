\documentclass[a4paper,12pt]{article}

\usepackage[slovak]{babel}
%\usepackage[T1]{fontenc}
\usepackage[IL2]{fontenc} % lepšia sadzba písmena Ľ než v T1
\usepackage[utf8]{inputenc}
\usepackage{graphicx}
\usepackage{url} % príkaz \url na formátovanie URL
\usepackage{hyperref} % odkazy v texte budú aktívne (pri niektorých triedach dokumentov spôsobuje posun textu)
\usepackage{float}
\usepackage{cite}
\usepackage{subcaption}
\usepackage{multicol}
\usepackage{atbegshi}
%\usepackage{times}



\title{Ako nás algoritmy ovplyvňujú: zber dát, personalizácia obsahu a ich dopad na spoločnosť}
\author{Jakub Žúbor}
\date{\today} % Dátum, alebo môžeš zadať konkrétny dátum

\begin{document}

% Titulná strana
\maketitle
\AtBeginShipoutNext{\AtBeginShipoutUpperLeft{
  \hspace{3cm}\raisebox{-3cm}{\includegraphics[width=4cm]{STU-FIIT-ncv.pdf}}}}
% Abstrakt
\begin{abstract}
Sociálne siete sa  v posledných rokoch stali populárnejšie ako kedykoľvek predtým. Za nárastom ich popularity stoja aj čoraz sofistikovanejšie algoritmy na zber a spracovanie informácií od užívateľov. Tieto algoritmy vytvárajú pre každého užívateľa jedinečný personalizovaný obsah na základe ich predošlých interakcií. Toto odporúčanie obsahu, ktorý je ušitý každému na mieru, môže viesť k vytvoreniu informačných bublín a podporovaniu rôznych stereotypov a ideológií ku ktorým konzumenti už predtým prejavovali sympatie, čo prispieva k polarizácii spoločnosti. 
V mojej práci sa budem venovať tomu, aké využite majú odporúčacie systémy na sociálnych sieťach, zberu dát a na čo sú tieto dáta využívané veľkými spoločnosťami, ako je napríklad Google, Facebook, Instagram, TikTok alebo X (v minulosti Twitter). Ďalej sa chcem takisto venovať dopadu odporúčacích systémov na spoločnosť a psychickú stránku jednotlivca (potenciál vzniku závislosti na sociálnych sieťach, FOMO – fear of missing out), a stratégiám, ktoré veľké spoločnosti používajú aby si získali a udržali našu pozornosť.
Bez odporúčacích systémov a algoritmov by sociálne siete nemohli fungovať a tým že aj naďalej narastá ich popularita sa odporúčacie systémy za pomoci AI a strojového učenia viac a viac zdokonaľujú a dopad odporúčacích systémov bude naďalej rásť.

\end{abstract}

\newpage

% Úvodná sekcia
\section{Úvod}
V tejto sekcii predstavíme tému článku a vysvetlíme, prečo je dôležitá.

% Druhá sekcia
\section{Metodológia}
Tu popíš metódy a prístupy, ktoré si použil/a na výskum alebo analýzu.

% Tretia sekcia
\section{Výsledky}
Výsledky tvojej práce, výskumu alebo experimentu budú zahrnuté v tejto sekcii.

% Štvrtá sekcia
\section{Diskusia}
V tejto časti môžeš diskutovať o výsledkoch, porovnávať ich s inými prácami alebo prístupmi.

% Piata sekcia
\section{Záver}
Krátke zhrnutie a závery, ktoré vyplývajú z tvojho článku.

% Literatúra
\begin{thebibliography}{9}
    \bibitem{ref1} Autor, \textit{Názov článku alebo knihy}, Rok vydania.
    % Pridaj ďalšie citácie podľa potreby
\end{thebibliography}

\end{document}